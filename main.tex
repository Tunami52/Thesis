\documentclass[a4paper,12pt]{memoir}

% \usepackage{lipsum}
% package for generating dummy text -- remove for actual thesis

\usepackage{amsmath,amssymb,amsthm} % standard AMS LaTeX packages

\usepackage{listings} % for typesetting programming code and similar elements
\lstset{
  columns=flexible,
  basicstyle={\small\ttfamily},
}

\usepackage{lu-thesis}
% load after amsmath!
% requires the memoir document class as well as the following packages:
% ebgaramond, ebgaramond-maths, graphicx, hyperref, newtxmath, xparse
% load with 'times' option to use a Times New Roman replacement as default font

\usepackage{xcolor} % access to more colours
\definecolor{lundblue}{RGB}{0,0,128}
\definecolor{lundbronze}{RGB}{156,97,20}

\usepackage[colorlinks]{hyperref}
% package for generating hyperlinks
% load after all other packages to avoid compatibility issues (unless a package
% complains and say you should do otherwise)
\hypersetup{%
  colorlinks=true,%
  linkcolor=lundblue,%
  urlcolor=lundblue,%
  citecolor=lundbronze,%
}

\usepackage{subfiles} 
% used to compile pdfs for sections seperately

%%%% BEGIN example usage of amsthm package

\numberwithin{equation}{section} % preface equation number with section number

% equations, theorems, propositions, lemmas and corollaries share the same counter
\theoremstyle{theorem}
\newtheorem*{utheorem}{Theorem} % unnumbered Theorem environment
\newtheorem{theorem}[equation]{Theorem} 
\newtheorem{proposition}[equation]{Proposition}
\newtheorem{lemma}[equation]{Lemma}
\newtheorem{corollary}[equation]{Corollary}

\theoremstyle{definition} % these environments are not typeset in italics, but
                          % the environment's name is typeset in boldface
\newtheorem*{udefinition}{Definition} % unnumbered Definition environment
\newtheorem{definition}[equation]{Definition}

\theoremstyle{remark} % these environments are not typeset in italics, and
                      % environment's name is typeset in italics instead of boldface
\newtheorem{remark}[equation]{Remark}
\newtheorem{example}[equation]{Example}

%%%% END example usage of amsthm package

\begin{document}

%%%%% BEGIN cover page

\author{Oskar Emmerich}
% the thesis' author

\title{Group Structure on arbitrary sets:\\An algebraic application of the Axiom of Choice}
% the thesis' title

\date{\today}
% the thesis' hand-in date

\maketitleLU{Bachelor's}{Anitha Thillaisundaram} % comment for master's thesis
% \maketitleLU[Master's][Advisor's name] % uncomment for master's thesis

%%%%% END cover page

\frontmatter % pages within the front matter are numbered using lowercase Roman numerals

%%%%% BEGIN abstract

\thispagestyle{empty}

\begin{abstract}
  The thesis should include an abstract that summarises its contents;
  mathematical jargon can be utilised here. The typical length of an abstract is
  between 100 and 300 words.
\end{abstract}

%%%%% END Abstract

%%%%% BEGIN popsci description

\chapter*{Popular science description}
\addcontentsline{toc}{chapter}{Popular science description}
\subfile{sections/popular_description.tex}

%%%%% END popsci description

%%%%% BEGIN toc

\cleardoublepage

\settocdepth{subsection}
\tableofcontents*

%%%%% END toc

%%%%% BEGIN intro

\chapter*{Introduction}
\addcontentsline{toc}{chapter}{Introduction}
\subfile{sections/introduction}

%%%%% END intro

%%%%% BEGIN main body of the thesis

\mainmatter % pages within the main matter are numbered using Arabic numerals

\chapter{First chapter}

%%%%% END main body of the thesis

%%%%% BEGIN appendices

\appendix % appendices are numbered using uppercase Roman letters

%%%%% END appendices

\backmatter

%%%%% BEGIN references

\bibliographystyle{alpha}
\bibliography{bibliography}

%%%%% END references

\end{document}
