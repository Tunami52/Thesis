\documentclass[../../main.tex]{subfiles}
\begin{document}
\section{Models of Formal Languages}
\begin{definition}\cite[Definition 1.1.1]{Mar02}
    A \textit{formal language} $\mathcal{L}$ in first order logic is given by the following:
    \begin{enumerate}
        \item A set $\mathcal{F}$ of functions $f$ of $n_f$ variables, with $n_f \in \mathbb{Z}^+$ a positive integer,
        \item A set $\mathcal{R}$ of $n_r$-ary relations $r$, with $n_r \in \mathbb{Z}^+$ a positive integer,
        \item A set $\mathcal{C}$ of constants.
    \end{enumerate}
\end{definition}

\section{The Löwenheim-Skolem Theorem}

\begin{lemma}\cite[Lemma 2.1.1]{Cha90}
    Let $T$ be a consistent set of sentences of $\mathcal{L}$.
    Let $C$ be a set of new constant symbols of power $\left\lvert C \right\rvert = \left\lVert \mathcal{L} \right\rVert$
     and let $\bar{\mathcal{L}} = \mathcal{L} \cup C$ be the simple expansion of $\mathcal{L}$ formed by adding $C$.
    
     Then $T$ can be expanded to a consistent set of sentences $\bar{T}$ in $\bar{\mathcal{L}}$, which has $C$ as a set of witnesses in $\bar{\mathcal{L}}$.
\end{lemma}

\begin{lemma}\cite[Lemma 2.1.2]{Cha90}
    Let $T$ be a set of sentences and let $C$ be a set of witnesses of $T$ in $\mathcal{L}$.
    Then $T$ has a model $\mathfrak{U}$, such that every element of $\mathfrak{U}$ is an interpretation of a constant $c \in C$.
\end{lemma}

\begin{theorem}[Extended Completeness Theorem]\cite[Theorem 1.3.21]{Cha90}
    Let $\Sigma$ be a set of sentences in $\mathcal{L}$.
    Then $\Sigma$ is consistent if and only if $\Sigma$ has a model.
\end{theorem}

\begin{theorem}[Downward Löwenheim-Skolem Theorem]\cite[Corollary 2.1.4]{Cha90}
    Every consistent theory $T$ in $\mathcal{L}$ has a model of power at most $\left\lVert \mathcal{L} \right\rVert$.
\end{theorem}

\begin{theorem}[Compactness Theorem]\cite[Theorem 1.3.22]{Cha90}
    A set of sentences $\Sigma$ has a model if and only if every finite subset of $\Sigma$ has a model.
\end{theorem}

\begin{theorem}[Upward Löwenheim-Skolem Theorem]\cite[Corollary 2.1.6]{Cha90}
    If $T$ has infinite models, then it has infinite models of any given power $\alpha \geq \left\lVert \mathcal{L} \right\rVert$.
\end{theorem}

\end{document}