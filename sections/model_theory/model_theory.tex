\documentclass[../../main.tex]{subfiles}
\begin{document}
The aim of this chapter is to give an introduction to the basic tools of model theory with which will then prove the \hyperref[up-lowenheim-skolem]{Löwenheim-Skolem Theorem}.
This theorem, divided into two parts, is dependent on \hyperref[choice-axiom]{The Axiom of Choice}, 
hence we will later use it to show that \textbf{AC} implies the existence of a group structure on all non-empty sets.

We were already concerned with axiomatizations in the previous chapters, however we will formalize these notions some more here.
The set theory axioms outlined in Chapter \ref{preliminaries} still hold true, 
but we will discover more about what is and is not true in for example group theory or other algebraic structures.
We start the treatment of model theory with the motivating example of ``Ehrenfeucht-Fra\"isse games'', 
exemplifying how the field relates to the linear orderings from Chapter \ref{orderings}.
The impatient or familiar reader however may skip this section without missing any necessary theory.

\section{Ehrenfeucht-Fra\"isse Games}
Let us suppose we are given two linear orderings, based on which we define a two player game.
Player I is called \textit{spoiler} and starts by picking a point on one of the two orderings, 
If they pick an element of $A$ we call this $a_1$ and if they pick an element of $B$ we call it $b_1$.
After the spoiler has picked their point, player $II$ will pick a point on the other linear ordering. 
We call player II \textit{replicator}.\footnote{The names ``spoiler'' and ``duplicator'' were coined by Joel Spencer.\cite[\S 6]{Wil24}
The literature \cite{Ros82} we mainly base this section of the text on uses ``Player I'' and ``Player II'' instead.}
Say spoiler picked $a_1$ in $A$, replicator then has to pick some element $b_1$ of $B$.

The spoiler and replicator go back and forth picking points on the linear orderings for a predetermined amount of turns $n$.
Spoiler always gets to choose the ordering they pick a point on and replicator has to pick something on the other ordering.
The second player, replicator, wins if the elements $a_1,\ldots,\, a_n$ are in the same order with respect to $A$ as the elements $b_1,\ldots,\, b_n$ are with respect to $B$.
Spoiler wins if replicator loses, if the elements are not in the same order. 
This is how a single play of an \textit{Ehrenfeucht-Fra\"isse game}, also called a \textit{Back-and-forth game}, in $n$ turns is played out.

\begin{example}
    We will go through the play of a game with $3$ steps. 
    Let $A = \mathbb{Q}$ and $B=\mathbb{Z}$ under the usual order be the two linear orderings that are played on.
    
    Spoiler starts and picks the point $a_1 = 0$ in $\mathbb{Z}$. Duplicator on their turn also picks the point $b_0 = 0$ but in $\mathbb{Z}$.
    On the next turn spoiler picks $a_2 = 1$ in $\mathbb{Z}$, duplicator again matches this and picks $b_2 = 1$ in $\mathbb{Q}$.
    Now however spoiler can and does pick $b_3$ in $\mathbb{Q}$ to be $\frac{1}{2}$. 
    This is not possible for duplicator to match since they are now confined to picking a point $a_3$ in $\mathbb{Z}$.
    Hence no matter what point duplicator picks to be $a_3$, they will lose.
\end{example}

%insert tikz image here%

\begin{example}
    Some example where duplicator has a winning strategy.
\end{example}

\begin{definition}[Ehrenfeucht-Fra\"isse Game]\cite[Definition 6.2]{Ros82}
    Formal definition of EF-Game.
\end{definition}

\begin{definition}[$G_n$-equivalence]\cite[Definition 6.8]{Ros82}
    Formal definition of EF-equivalence.
\end{definition}

\begin{theorem}\cite[Theorem 2.4.6]{Mar02}\label{elementary-eq-th}
    Let $\mathcal{M}$ and $\mathcal{N}$ be two $\mathcal{L}$-structures, where $\mathcal{L}$ is a finite language without any function symbols.
    Then duplicator has a winning strategy in $G_n(\mathcal{M},\, \mathcal{N})$ for all $n \in \mathbb{N}$ if and only if $\mathcal{M}$ and $\mathcal{N}$ satisfy exactly the same theories.
\end{theorem}

We will not prove Theorem \ref{elementary-eq-th} here, but rather include it as an appetizer;
a means show the usefulness of model theory applied to problems we may already be interested in.
The reader interested in the proof of the theorem may read it in the sub-subsection titled ``Ehrenfeucht-Fra\"isse Games'' in \cite[\S 2.4]{Mar02}.
The rest of this chapter should provide an adequate, although maybe not ideal, background to understanding it.

\section{Models of Formal Languages}
\begin{definition}\cite[Definition 1.1.1]{Mar02}
    A \textit{formal language} $\mathcal{L}$ in first order logic is given by the following:
    \begin{enumerate}
        \item A set $\mathcal{F}$ of functions $f$ of $n_f$ variables, with $n_f \in \mathbb{Z}^+$ a positive integer,
        \item a set $\mathcal{R}$ of $n_R$-ary relations $R$, with $n_R \in \mathbb{Z}^+$ a positive integer,
        \item a set $\mathcal{C}$ of constants.
    \end{enumerate}
\end{definition}

Since we utilize set theory in the definition of languages, we consider the set inclusion symbol $\in$ to be to be part of first-order logic here (in the same vein as $=$.)
As such, when we dealt with set theoretic constructions we used the language of given by $\mathcal{L}_{Set} = \varnothing$.
Some other examples of languages are those of groups or of rings,
\begin{align*}
    \mathcal{L}_g = \left\{\cdot,\, e\right\} 
    && \text{and} &&
    \mathcal{L}_r = \left\{+,\,\cdot,\, 0,\, 1\right\},
\end{align*}
where $+$, and $\cdot$ are binary functions and $e$, $0$, $1$ are constants.

These are just \textit{languages} however, they do not dictate any additional structure related to the objects they contain.
The axioms of groups and rings are not part of the language itself, for groups, say, we only know the parity of $\cdot$ and that some distinguished element $e$ exists.
As such the language of rings is also the same as the language of fields, since fields are just a special case of rings.

The existence of constant symbols also means we cannot just reuse the definition of a formula we used for sets.
As such we first define \textit{terms}, a notion of chaining together operations.

\begin{definition}\cite[Definition 1.1.4]{Mar02}
    We say that $T$ is the set of $\mathcal{L}$-terms, if $T$ is the smallest set such that
    \begin{enumerate}
        \item $c \in T$ for each constant $c \in \mathcal{C}$
        \item $v_i \in T$ for variable symbols $v_i$, where $i = 1,\, 2,\ldots$,
        \item $f(t_1,\ldots,\, t_n) \in T$ for $f \in \mathcal{F}$ and terms $t_1,\ldots,\, t_n \in T$.
    \end{enumerate}
\end{definition}

\begin{definition}[Formula of a Set]
    An \textit{atomic formula} in a language $\mathcal{L}$ is either
    \begin{enumerate}
        \item $t_1 = t_2$ for terms $t_1,\, t_2 \in T$, or   
        \item $R(t_1,\ldots,\, t_{n_R})$ for $R \in \mathcal{R}$ and terms $t_1,\ldots,\, t_{n_R} \in T$.
    \end{enumerate}
    A \textit{formula} $\phi$ in $\mathcal{L}$ is an any combination of atomic formulas with logical connectives and quantifiers.
\end{definition}

In order to actually utilize languages we also define the namesake of model theory, models.

\begin{definition}\cite[Definition 1.1.2]{Mar02}
    We $\mathcal{L}$-\textit{structure} or \textit{model} $\mathcal{M}$ is given by the following:
    \begin{enumerate}
        \item A nonempty set $M$,
        \item a function $f^{\mathcal{M}}: M^{n_f} \to M$ for each $f \in \mathcal{F}$,
        \item a set $R^\mathcal{M} \subseteq M^{n_R}$ for each $R \in \mathcal{R}$,
        \item an element $c^\mathcal{M} \in M$ for each $c \in \mathcal{C}$.
    \end{enumerate}
    The set $M$ is referred to as the \textit{universe}, \textit{domain} or \textit{underlying set} of $\mathcal{M}$ 
    and $f^{\mathcal{M}}$, $R^{\mathcal{M}}$ and $c^{\mathcal{M}}$ are called the \textit{interpretations} of $\mathcal{M}$.
    We sometimes also identify a model $\mathcal{M}$ with the pair $\left<M,\, \mathcal{I}\right>$, 
    where $\mathcal{I}$ is the function mapping $\mathcal{F}$, $\mathcal{R}$ and $\mathcal{C}$ to their respective interpretations in $M$.
\end{definition}

\section{The Löwenheim-Skolem Theorem}

\begin{lemma}\cite[Lemma 2.1.1]{Cha90}
    Let $T$ be a consistent set of sentences of $\mathcal{L}$.  
    Let $C$ be a set of new constant symbols of power $\left\lvert C \right\rvert = \left\lVert \mathcal{L} \right\rVert$
     and let $\bar{\mathcal{L}} = \mathcal{L} \cup C$ be the simple expansion of $\mathcal{L}$ formed by adding $C$.
    
    Then $T$ can be expanded to a consistent set of sentences $\bar{T}$ in $\bar{\mathcal{L}}$, which has $C$ as a set of witnesses in $\bar{\mathcal{L}}$.
\end{lemma}

\begin{lemma}\cite[Lemma 2.1.2]{Cha90}
    Let $T$ be a set of sentences and let $C$ be a set of witnesses of $T$ in $\mathcal{L}$.
    Then $T$ has a model $\mathfrak{U}$, such that every element of $\mathfrak{U}$ is an interpretation of a constant $c \in C$.
\end{lemma} 

\begin{theorem}[Extended Completeness Theorem]\cite[Theorem 1.3.21]{Cha90}
    Let $\Sigma$ be a set of sentences in $\mathcal{L}$.
    Then $\Sigma$ is consistent if and only if $\Sigma$ has a model.
\end{theorem}

\begin{theorem}[Downward Löwenheim-Skolem Theorem]\label{down-lowenheim-skolem}\cite[Corollary 2.1.4]{Cha90}
    Every consistent theory $T$ in $\mathcal{L}$ has a model of power at most $\left\lVert \mathcal{L} \right\rVert$.
\end{theorem}

\begin{theorem}[Compactness Theorem]\cite[Theorem 1.3.22]{Cha90}
    A set of sentences $\Sigma$ has a model if and only if every finite subset of $\Sigma$ has a model.
\end{theorem}

\begin{theorem}[Upward Löwenheim-Skolem Theorem]\label{up-lowenheim-skolem}\cite[Corollary 2.1.6]{Cha90}
    If $T$ has infinite models, then it has infinite models of any given power $\alpha \geq \left\lVert \mathcal{L} \right\rVert$.
\end{theorem}

\end{document}