\documentclass[../../main.tex]{subfiles}
\begin{document}

\subsection*{Histrical Backround}

In 1902 Bertrand Russel showed with what is now known as \textit{Russel's Paradox} that the previously used approach to set theory was inconsitent.
Ernst Zermelo then created an axiomatic framework for set theory in 1905, motivated both by attempting to preserve results such as the thoery of infinities by Georg Cantor, as well as avoiding paradoxes.
These axioms, later modified by Abraham Fraenkel, became known as the nine \textit{Zermelo-Fraenkel Axioms} (ZF) as well as the \textit{Axiom of Choice} (AC)\cite[pp.66-70, 75]{Gol17}.

The axiom of choice in particular is of special interest in many areas of mathematics, especially in algebra and topology, ofthen in the form of the equivalent statement of \textit{Zorn's Lemma},
which says that every non-emtpy partially ordered set with an upper bound has a maximal element \cite{Jec78}.

Finally in 1971 András Hajnal and Andor Kertész published a paper \cite{Haj72} which provided another equivalence to AC, namely that there exists a cancellative groupoid structure on every (uncountably infinite) set.
This paper makes use of first-order model theory, an area of logic developed during the first half of the 20th century, which utalises models of formal languages to obtain results.
Kertész later expanded on this, providing an alternative algebraic partial proof in a lecture series given at the University of Jyväskylä \cite{Ker75}.

\subsection*{Thesis Structure}

The aim of this thesis is to provide context to the paper \cite{Haj72} and to derive the theory needed for the proof of its main theorem:
\begin{theorem}
    The following sentences are equivalent in ZF:
    \begin{enumerate}
        \item Axiom of Choice
        \item Every non-empty set admits a cancellative groupoid structure
    \end{enumerate}
\end{theorem}

We will first explore orderings and well-orderings in the context of axiomatic set theory.
Of special importance here will be Zorn's Lemma, a well-known equivalence of AC.
We will finish by giving a proof for a lemma by Hartogs \cite{Har15}, which states that there always exists an ordinal to which no subset of an arbitrary set can be injectively mapped into.

We will then move on to an introduction to model theory, with the aim of proving the upwards Löwenheim-Skolem Theorem.
Model theory is a very useful tool for applying results from logic to non-logic areas of mathematics, especially abstract algebra as we will see later.
As Chang and Heisler put it in \cite{Cha90} (a very good historical introduction to model theory and the first comprehensive textbook for the subject), $$\textbf{Model Theory = Universal Algebra + Logic}.$$

Finally we will give a detailed proof of the aforementioned theorem by Hajnal and Kertész, applying the results by Hartogs and Löwenheim and Skolem.
\end{document}