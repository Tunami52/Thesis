\documentclass[../../main.tex]{subfiles}
\begin{document}

In 1902 Bertrand Russel showed with what is now known as \textit{Russel's Paradox} that the previously used approach to set theory was inconsitent.
Ernst Zermelo then created an axiomatic framework for set theory in 1905, motivated both by attempting to preserve results such as the thoery of infinities by Georg Cantor, as well as avoiding paradoxes.
These axioms, later modified by Abraham Fraenkel, became known as the nine \textit{Zermelo-Fraenkel Axioms} (ZF) as well as the \textit{Axiom of Choice} (AC)\cite[pp.66-70, 75]{6350139}.

The axiom of choice in particular is of special interest in many areas of mathematics, especially in algebra and topology, ofthen in the form of the equivalent statement of \textit{Zorn's Lemma},
which says that every non-emtpy partially ordered set with an upper bound has a maximal element \cite{3110983}.

Finally in 1971 András Hajnal and Andor Kertész published a paper \cite{MR329895} which provided another equivalence to AC, namely that there exists a cancellative groupoid structure on every (uncountably infinite) set.
This paper makes use of first-order model theory, an area of logic developed during the first half of the 20th century, which utalises models of formal languages to obtain results.
Kertész later expanded on this, providing an alternative algebraic partial proof in a lecture series given at the University of Jyväskylä \cite{4993767}.

\end{document}