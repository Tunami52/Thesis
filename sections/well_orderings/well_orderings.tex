\documentclass[../../main.tex]{subfiles}
\begin{document}

\section{Linear, Partial and Well-orderings}

\section{Ordinals and Order Types}

\section{The Well-ordering Theorem}
The following, along with \textit{Zorn's Lemma}, is one of the most fundamental results in set theory.
There is a (bad) joke that goes:
\begin{quote} %find source
    The \textit{Axiom of Choice} is obviously true, the \textit{well-ordering theorem} obviously false, 
    and who knows with \textit{Zorn's Lemma}.
\end{quote}
If you are not dying of laughter already, I am sure you will by the end of this section.

\begin{definition}[Zermelo's Well-ordering Theorem]\cite[Theorem 15]{Jec78}\label{well-order-thm}
    Every set can be well ordered.
\end{definition}

We could provide a proof for \ref{well-order-thm} in \textbf{ZFC} here, 
and treat the well-ordering theorem as a regular theorem.
This theorem, as it turn out, is not just another regular theorem, and we will therefore also not treat it as one.

Indeed, the well-ordering theorem is actually equivalent to the Axiom of Choice.
This means that if either statement is assumed to be true (and it has to be assumed since we are talking about \textit{axioms}),
the other one can be proved from it.
This is also how we will proceed with the proof, first showing that the well-ordering theorem is true in \textbf{ZFC},
then conversely proving \textbf{AC} in \textbf{ZF}, assuming that the well-ordering theorem holds true.

\begin{theorem}
    \hyperref[well-order-thm]{The Well-ordering Theorem} is equivalent to the Axiom of Choice. %cross-reference AC definition
\end{theorem}

\section{Hartogs' Lemma}
We continue with the final result for this chapter, 
a lemma originally stated by Hartogs in 1915, restated in this form in our main paper \cite{Haj72}.

\begin{lemma}\cite{Har15}
    Let $A$ be an arbitrary set.
    Then there exists an ordinal $\alpha$, 
    %such that no subset of $A$ can be injectively mapped on to $\alpha$.
    such that no injective map from any subset of $A$ to $\alpha$ exists. %is this correct? it makes no intuitive sense...
\end{lemma}

\end{document}