\documentclass[../../main.tex]{subfiles}
\begin{document}

\section{Linear, Partial and Well-Orderings}

\section{Ordinals and Order Types}
There are several ways to define the natural numbers $\mathbb{N}$, the way we do it here, and the way generally used in set theory, is to use the \hyperref[ZF7]{Axiom of Infinity}.
This states that $\mathbb{N}$ is an \textit{inductive set}, meaning that it contains $0$, defined as $\varnothing = \left\{\right\}$, as well as the successor of every element in it, including of course $0$ itself. \cite[p.39]{Gol17}

\begin{definition}\cite[p.38]{Gol17}
    The successor of a set $\alpha$ is $\alpha^+ = \alpha \cup \left\{\alpha\right\}$.
    The successor of $0 = \varnothing$ is called $1$ and the successor of $1$ is called $2$, etc.. 
\end{definition}

The consequence of this is that $\mathbb{N}$ is the set we are familiar with: $\left\{0,\, 1,\, 2,\, 3,\ldots\right\}$.
It also means that any natural number is defined as the set of all of its predecessors.
For example $3 = \left\{0,\, 1,\, 2\right\}$ and $5 = \left\{0,\, 1,\, 2,\, 3,\, 4\right\}$.

Perhaps slightly more subtle then is that, under the usual ordering, $n < m \implies n \in m \wedge n \subset m$.
This is an important property and the natural numbers, as well as the set $\mathbb{N}$ at large, are \textit{transitive} sets.

\begin{definition}\cite[p.14]{Jec78}
    A set $T$ is called \textit{transitive} if $$\forall x \left(x \in T \implies x \subseteq T\right).$$
\end{definition}

\begin{definition}\cite[p.14]{Jec78}
    A set is called an \textit{ordinal number} or \textit{ordinal} if it is transitive and well-ordered by $\in$.
\end{definition}

Ordinals are denoted by lowercase greek letter: $\alpha,\, \beta,\, \gamma,\ldots.$
The ordinal associated with $\left(\mathbb{N},\, \in\right)$ specifically is denoted by $\omega$.

\begin{lemma}
    $\omega$ and every natural number $n \in \omega$ is an ordinal.
\end{lemma}

\begin{lemma}\cite[Lemma 2.3, p.15]{Jec78}
    \begin{enumerate}
        \item If $\alpha$ is an ordinal and $\beta \in \alpha$, then $\beta$ is an ordinal.
        \item If $\alpha,\, \beta$ are ordinals and $\alpha \subset \beta$, then $\alpha \in \beta$.
        \item If $\alpha,\, \beta$ are ordinals, then either  $\alpha \subseteq \beta$ or $\beta \subseteq \alpha$
    \end{enumerate}
\end{lemma}

\begin{theorem}\cite[Theorem 2, p.15]{Jec78}
    Every well-ordered set is order isomorphic to a unique ordinal.
\end{theorem}

\section{The Well-Ordering Theorem}
The following, along with \textit{Zorn's Lemma}, is one of the most fundamental results in set theory.
There is a (bad) joke that goes:
\begin{quote} %find source
    The \textit{Axiom of Choice} is obviously true, the \textit{Well-Ordering Theorem} obviously false, 
    and who knows with \textit{Zorn's Lemma}.
\end{quote}

\begin{definition}[Zermelo's Well-Ordering Theorem]\cite[Theorem 15, p.39]{Jec78}\label{well-order-thm}
    \newline Every set can be well ordered.
\end{definition}

We could provide a proof for definition \ref{well-order-thm} in \textbf{ZFC} here directly.
This theorem, as it turn out, is not just another regular theorem, and we will therefore also not treat it as one.

Indeed, the Well-Ordering Theorem is actually equivalent to the Axiom of Choice.
This means that if either statement is assumed to be true (and it has to be assumed since we are talking about \textit{axioms}),
the other one can be proved from it.
This is the same methodology we will use for proving our main result, theorem \ref{main-thm}, as well.
There we will show equivalence of our main statement, that a group structure exists on all arbitrary sets, with the Well-Ordering Theorem.
As such by transitivity, this main statement is also equivalent to the Axiom of Choice.

\begin{theorem}
    \hyperref[well-order-thm]{The Well-ordering Theorem} is equivalent to the Axiom of Choice. %cross-reference AC definition
\end{theorem}

\begin{proof}\cite[Theorem 15, p.39]{Jec78}
    We provide a proof in two parts; first showing that the Well-Ordering Theorem is true in \textbf{ZFC}.
    Then, conversely, we prove \textbf{AC} in \textbf{ZF}, assuming that the Well-Ordering Theorem holds true.
    \begin{enumerate}
        \item \textbf{Axiom of Choice} $\implies$ \textbf{Well-Ordering Theorem}
        
        We proceed by transfinite induction.

        Let $A$ be an arbitrary set and let $S = \mathcal{P}(A) \setminus \varnothing$ be the collection of all non-empty subsets of $A$.
        Let $f: S \to A$ be a choice function (as specified by the Axiom of Choice).
        We then define an ordinal sequence $\left(a_{\alpha} \, \vert \, \alpha < \theta \right)$ the following way:
        \begin{align*}
            a_0 &= f(A)\\
            a_{\alpha} &= f\left(A \setminus \left\{a_\xi \, \vert \, \xi < \alpha\right\} \right)
            &\text{if}\ A \setminus \left\{a_\xi \, \vert \, \xi < \alpha\right\}\ \text{is non-empty}.
        \end{align*}
        Now let $\theta$ be the smallest ordinal such that $A = \left\{a_\xi \, \vert \, \xi < \theta\right\}$.
        
        We know that such an ordinal must exist, since the sequence $\left(a_{\alpha} \, \vert \, \alpha < \theta \right)$ is entirely defined by the choice function $f$.
        The function $f$ maps every non-empty subset of $A$, i.e. members of $S$, to an element of that subset (in $A$).
        
        By defining the ordinal sequence the way we did, it is not possible for any element of $A$ to occur in the sequence twice.
        Any subset of $A$, which is the input of the choice function for some element $a_\gamma$ in the sequence, does not contain any elements $a_\alpha$ for $\alpha < \gamma$, and by definition $f$ cannot map to any of these members.
        
        As such $\mathbf{Im}\left(\left(a_{\alpha} \, \vert \, \alpha < \theta \right)\right) = A$
        and $\left(a_{\alpha} \, \vert \, \alpha < \theta \right)$ enumerates $A$, meaning the sequence is a bijection.\footnote{Recall that a sequence is just a function from $\mathbb{N}$, respectively an ordinal, to the set of its elements}
        Hence $A$ can be well-ordered, the least element of any subset being the one which corresponds to the smallest ordinal in the sequence.
        \item \textbf{Well-Ordering Theorem} $\implies$ \textbf{Axiom of Choice}
        
        Let $S$ be a set of non-empty sets.

        The union $\bigcup S$ can be well-ordered by assumption and clearly $s \in S$ implies $s \subseteq \bigcup S$.
        We can then define the function $f: S \to \bigcup S$ to map any elements of $S$ to its least element, according to the well-order of $\bigcup S$.
        
        Evidently $f$ is a choice function and since the set $S$ was arbitrary the Axiom of Choice holds. \qedhere
    \end{enumerate}
\end{proof}

\section{Hartogs' Lemma}
We continue with the final result for this chapter, 
a lemma originally stated by Hartogs in 1915, restated in this form in our main paper \cite{Haj72}.

\begin{lemma}\cite{Har15}
    Let $A$ be an arbitrary set.
    Then there exists an ordinal $\alpha$, 
    %such that no subset of $A$ can be injectively mapped on to $\alpha$.
    such that no injective map from any subset of $A$ to $\alpha$ exists. %is this correct? it makes no intuitive sense...
\end{lemma}

\end{document}