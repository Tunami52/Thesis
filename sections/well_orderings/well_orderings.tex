\documentclass[../../main.tex]{subfiles}
\begin{document}

\section{Linear, Partial and Well-Orderings}

We start by defining the two types of partial orderings, \textit{strict} and \textit{weak} ones.
To give a more intuitive understanding of how these differ we use the notation $<$ and $\leq$ respectively, but $R$ is also commonly used to denote a relation.

\begin{definition}[Strict Partial Order]\cite[p.165]{Gol17}
    Let $X$ be a set and $<\, \subseteq X \times X$ a binary relation on $X$.
    Then $<$ is called a \textit{(strict) partial order of $X$}, and $(X, <)$ called a \textit{(strictly) partially ordered set}, if it is
    \begin{enumerate}[label=(\roman*)]
        \item \textbf{irreflexive}: $\forall x \in X \left(x \not < x\right)$
        \item \textbf{transitive}: $\forall x,\, y,\, z \in X \left(\left(x < y \wedge y < z\right) \implies x < z\right)$
    \end{enumerate}
    It is called \textit{linear} if for all $x$, $y$ in $X$, $x < y$ or $y < x$ or $x = y$.
\end{definition}

\begin{definition}[Weak Partial Order]\cite[p.164]{Gol17}
    Let $X$ be a set and $\leq\, \subseteq X \times X$ a binary relation on $X$.
    Then $\leq$ is called a \textit{weak partial order} of $X$, and $(X, \leq)$ called a \textit{weakly partially ordered set}, if it is
    \begin{enumerate}[label=(\roman*)]
        \item \textbf{reflexive}: $\forall x \in X \ (x \leq x)$
        \item \textbf{transitive}: $\forall x,\, y,\, z \in X \left(\left(x \leq y \wedge y \leq z\right) \implies x \leq z\right)$
        \item \textbf{anti-symmetric}: $\forall x,\, y \in X \left((x \leq y) \wedge (y \leq x) \implies x = y\right)$
    \end{enumerate} 
    It is called \textit{linear} if for all $x$, $y$ in $X$, $x \leq y$ or $y \leq x$.
\end{definition}

\begin{definition}\cite[p.13]{Jec78}
    If $\left(X,\, <_X\right)$ and $\left(Y,\, <_Y\right)$ are two partially ordered sets, we call a function $f: X \to Y$ \textit{order-preserving} if
    $$x_1 <_X x_2 \iff f(x_1) <_Y f(x_2).$$
    If $X$ and $Y$ are both linearly ordered, an order-preserving function is also said to be \textit{increasing}.

    The function $f$ is called an \textit{order-isomorphism} if $f$ is both order-preserving and a bijective.
    Whenever it is clear from context that we are talking about ordered sets we simply call $f$ an \textit{isomorphism}.
    If $f$ is order-preserving and injective, it is called an \textit{order-embedding}. \cite[p.167]{Gol17}
\end{definition}

A partially ordered set $(X, <)$ is sometimes also referred to simply as $X$ by some abuse of notation when the relation $<$ is known.
Additionally, whenever we talk about partially or linearly ordered sets without specifying which type, and where the type of partial order matters, we are referring to strict ones. \cite[p.12]{Jec78}
Out of convenience, when talking about a strict partial order $<$, we sometimes refer to the term $\left(a < b \vee a = b\right)$ as $a \leq b$.

Clearly it is straightforward to define a weak partial order $R'$ from a strict partial order $R$, letting $\left<x,\, y\right> \in R'$ whenever $\left<x,\, y\right> \in R'$ or $x = y$.

\begin{definition}\cite[p.12]{Jec78}
    An element $a$ of an ordered set $\left(X, <\right)$ is the \textit{least element} of $X$ with respect to $<$, if $\forall x \in X  \left(a < x \vee x = a\right)$.
    Similarly, an element $z$ is called the \textit{greatest element} of $X$ if $\forall x \in X  \left(x < a \vee x = a\right)$.
\end{definition}

This notion of a least element lets us define a special kind of linearly ordered set:
\begin{definition}[Well-Order]\cite[p.13]{Jec78}
    A strict linear order $<$ of a set $X$ is called a \textit{well-ordering} if every subset of $X$ has a least element.
\end{definition}

Well-ordered sets are central to the axiomatic set theory at hand.
In fact, one of the most important results we will treat here, is that every set can we well-ordered (with the Axiom of Choice).

Further, we will introduce the concept of ordinals as a way effectively classify all well-ordered sets.
The next two lemmata are needed for the proof of Theorem \ref{unique-ordinal}, an important result with regards to ordinals.
For this we need to define the initial section of an ordered set first:

\begin{definition}
    If $X$ is a well-ordered set and $s \in X$, we call the set $\left\{x \in X \,\vert\, x < s\right\}$ an \textit{initial segment} of $X$.
\end{definition}

\begin{lemma}\label{increasing-fcn-lemm}\cite[Lemma 2.1, p.13]{Jec78}
    If $\left(W,\, <\right)$ is a well-ordered set and $f: W \to W$ is an increasing function, then $f(x) \geq x$ for each $x\in W$.
\end{lemma}

\begin{proof}\cite[Lemma 2.1, p.13]{Jec78}
    In order to contrive a contradiction, we assume that $X = \left\{x \in W \,\vert\, f(x) < x\right\}$, the collection of elements of $W$ not satisfying the lemma, is a non-empty set.
    We then let $z$ be the least element of $X$ and $w = f(z)$ its preimage in $f$.
    By the definition of $X$ we this means that $f(w) < w$, contradicting the initial assumption that $f$ is an increasing function.
\end{proof}

\begin{lemma}\cite[Lemma 2.2, p.13]{Jec78}
    No well-ordered set is isomorphic to an initial segment of itself.
\end{lemma}

\begin{proof}\cite[Lemma 2.2, p.13]{Jec78}
    Assume for a contradiction that $f$ is an order isomorphism from an ordered set $\left(X,\, <\right)$ to an initial segment $\left(S,\, <\right) = \left\{x \in X \,\vert\, x < s\right\}$, for some $s \in X$ of itself.
    The image of $f$ is then $\mathbf{Im}\left(f\right) = \left\{x \in X \,\vert\, x < s\right\} = S$, but we know this is not possible by Lemma \ref{increasing-fcn-lemm}. 
\end{proof}

We are now ready to define ordinal numbers, motivated by the way we describe the natural numbers in set theory.

\section{Ordinals and Order Types}\label{ordinals-section}
There are several ways to define the natural numbers $\mathbb{N}$, the way we do it here, and the way generally used in set theory, is to use the \hyperref[ZF7]{Axiom of Infinity}.
This states that $\mathbb{N}$ is an \textit{inductive set}, meaning that it contains $0$, defined as $\varnothing = \left\{\right\}$, as well as the successor of every element in it, including of course $0$ itself. \cite[p.39]{Gol17}

\begin{definition}\cite[p.38]{Gol17}
    The successor of a set $\alpha$ is $\alpha^+ = \alpha \cup \left\{\alpha\right\}$.
    The successor of $0 = \varnothing$ is called $1$ and the successor of $1$ is called $2$, etc.
\end{definition}

The consequence of this is that $\mathbb{N}$ is the set we are familiar with: $\left\{0,\, 1,\, 2,\, 3,\ldots\right\}$.
It also means that any natural number is defined as the set of all of its predecessors.
For example $3 = \left\{0,\, 1,\, 2\right\} = \left\{\left\{\right\},\, \left\{\left\{\right\}\right\},\, \left\{\left\{\right\},\, \left\{\left\{\right\}\right\}\right\}\right\}$ and $5 = \left\{0,\, 1,\, 2,\, 3,\, 4\right\}$.

Perhaps slightly more subtle then is that, under the usual ordering, $n < m \implies n \in m \wedge n \subset m$.
This is an important property and the natural numbers, as well as the set $\mathbb{N}$ at large, are called \textit{transitive} sets.
The notion of a \textit{transitive set} is not to be confused with that of a \textit{transitive (binary) relation}, which is an unfortunate overlap in terminology.

\begin{definition}\cite[p.14]{Jec78}
    A set $T$ is called \textit{transitive} if $$\forall x \left(x \in T \implies x \subseteq T\right).$$
\end{definition}

\begin{definition}\cite[p.14]{Jec78}
    A set is called an \textit{ordinal number} or \textit{ordinal} if it is transitive and well-ordered by $\in$.
    We say $\alpha < \beta$ if and only if $\alpha \in \beta$.
\end{definition}

Ordinals are denoted by lowercase Greek letters: $\alpha,\, \beta,\, \gamma,\ldots.$
The ordinal associated with $\left(\mathbb{N},\, \in\right)$ specifically is denoted by $\omega$.
We know that $\omega$ is indeed an ordinal by construction.
It follows from the following lemma that every natural number also is an ordinal with respect to set inclusion.

\begin{lemma}\cite[Lemma 2.3, p.15]{Jec78}
    \begin{enumerate}
        \item The empty set $\varnothing$ is an ordinal.
        \item If $\alpha$ is an ordinal and $\beta \in \alpha$, then $\beta$ is an ordinal.
        \item If $\alpha,\, \beta$ are ordinals and $\alpha \subset \beta$, then $\alpha \in \beta$.
        \item If $\alpha,\, \beta$ are ordinals, then either  $\alpha \subseteq \beta$ or $\beta \subseteq \alpha$.
    \end{enumerate}
\end{lemma}

\begin{proof}\cite[Lemma 2.3, p.15]{Jec78}
    \begin{enumerate}
        \item The empty set has no non-empty subsets, hence it is transitive and well-ordered by $\in$.
        \item If $\beta \in \alpha$, then $\beta \subseteq \alpha$ by definition. Since $\alpha$ is well-ordered and transitive, so is $\beta$.
        \item Let $\gamma$ be the least element of the set $\beta \setminus \alpha$. We show that $\alpha = \gamma$.
        
        The ordinal $\alpha$ is transitive by definition and from this it follows that there are no ``gaps'' in the order. Indeed $\alpha$ must be an initial segment of $\beta$.
        As an initial segment, we can describe $\alpha$ as the set $\left\{\xi \in \beta \,\vert\, \xi < \gamma\right\}$.
        Again by the definition of ordinals, this is the set $\gamma$ itself and $\alpha = \beta$.
        \item We know that the intersection $\alpha \cap \beta = \gamma$ must be an ordinal, since not least the empty set also is an ordinal.
        However anything other than $\alpha = \gamma$ or $\beta = \gamma$ results in a contradiction:
        
        Assume for this contradiction that $\gamma \in \alpha$. Then $\gamma \in \beta$ by the second point of the lemma. 
        Because $\gamma$ is defined as the intersection of $\alpha$ and $\beta$, this means that $\gamma \in \gamma$.
        Since $\gamma$ is an ordinal, strictly linearly ordered, this is not possible. \qedhere
    \end{enumerate}
\end{proof}

\begin{theorem}\label{unique-ordinal}\cite[Theorem 2, p.15]{Jec78}
    Every well-ordered set is order isomorphic to a unique ordinal.
\end{theorem}

\section{The Well-Ordering Theorem}
The following, along with \textit{Zorn's Lemma}, is one of the most fundamental results in set theory.
There is a (bad) joke that goes:
\begin{quote} %find source
    The \textit{Axiom of Choice} is obviously true, the \textit{Well-Ordering Theorem} obviously false, 
    and who knows with \textit{Zorn's Lemma}.
\end{quote}

\begin{definition}[Zermelo's Well-Ordering Theorem]\cite[Theorem 15, p.39]{Jec78}\label{well-order-thm}
    \newline Every set can be well ordered.
\end{definition}

We could provide a proof for Definition \ref{well-order-thm} in \textbf{ZFC} here directly.
This theorem, as it turn out, is not just another regular theorem, and we will therefore also not treat it as one.

Indeed, the Well-Ordering Theorem is actually equivalent to the Axiom of Choice.
This means that if either statement is assumed to be true (and it has to be assumed since we are talking about \textit{axioms}),
the other one can be proved from it.
This is the same methodology we will use for proving our main result, theorem \ref{main-thm}, as well.
There we will show equivalence of our main statement, that a group structure exists on all arbitrary sets, with the Well-Ordering Theorem.
As such by transitivity, this main statement is also equivalent to the Axiom of Choice.

\begin{theorem}
    \hyperref[well-order-thm]{The Well-ordering Theorem} is equivalent to the Axiom of Choice. %cross-reference AC definition
\end{theorem}

\begin{proof}\cite[Theorem 15, p.39]{Jec78}
    We provide a proof in two parts; first showing that the Well-Ordering Theorem is true in \textbf{ZFC}.
    Then, conversely, we prove \textbf{AC} in \textbf{ZF}, assuming that the Well-Ordering Theorem holds true.
    \begin{enumerate}
        \item \textbf{Axiom of Choice} $\implies$ \textbf{Well-Ordering Theorem}
        
        We proceed by transfinite induction.

        Let $A$ be an arbitrary set and let $S = \mathcal{P}(A) \setminus \varnothing$ be the collection of all non-empty subsets of $A$.
        Let $f: S \to A$ be a choice function (as specified by the Axiom of Choice).
        We then define an ordinal sequence $\left(a_{\alpha} \, \vert \, \alpha < \theta \right)$ the following way:
        \begin{align*}
            a_0 &= f(A)\\
            a_{\alpha} &= f\left(A \setminus \left\{a_\xi \, \vert \, \xi < \alpha\right\} \right)
            &\text{if}\ A \setminus \left\{a_\xi \, \vert \, \xi < \alpha\right\}\ \text{is non-empty}.
        \end{align*}
        Now let $\theta$ be the smallest ordinal such that $A = \left\{a_\xi \, \vert \, \xi < \theta\right\}$.
        
        We know that such an ordinal must exist, since the sequence $\left(a_{\alpha} \, \vert \, \alpha < \theta \right)$ is entirely defined by the choice function $f$.
        The function $f$ maps every non-empty subset of $A$, i.e. members of $S$, to an element of that subset (in $A$).
        
        By defining the ordinal sequence the way we did, it is not possible for any element of $A$ to occur in the sequence twice.
        Any subset of $A$, which is the input of the choice function for some element $a_\gamma$ in the sequence, does not contain any elements $a_\alpha$ for $\alpha < \gamma$, and by definition $f$ cannot map to any of these members.
        
        As such $\mathbf{Im}\left(\left(a_{\alpha} \, \vert \, \alpha < \theta \right)\right) = A$
        and $\left(a_{\alpha} \, \vert \, \alpha < \theta \right)$ enumerates $A$, meaning the sequence is a bijection.\footnote{Recall that a sequence is just a function from $\mathbb{N}$, respectively an ordinal, to the set of its elements}
        Hence $A$ can be well-ordered, the least element of any subset being the one which corresponds to the smallest ordinal in the sequence.
        \item \textbf{Well-Ordering Theorem} $\implies$ \textbf{Axiom of Choice}
        
        Let $S$ be a set of non-empty sets.

        The union $\bigcup S$ can be well-ordered by assumption and clearly $s \in S$ implies $s \subseteq \bigcup S$.
        We can then define the function $f: S \to \bigcup S$ to map any elements of $S$ to its least element, according to the well-order of $\bigcup S$.
        
        Evidently $f$ is a choice function and since the set $S$ was arbitrary the Axiom of Choice holds. \qedhere
    \end{enumerate}
\end{proof}

\section{Hartogs' Lemma}
We continue with the final result for this chapter, 
a lemma originally stated by Hartogs in 1915, restated and proven in this form in our main paper by Hajnal and Kertész.

\begin{lemma}\cite{Har15}
    Let $A$ be an arbitrary set and let $S = \mathcal{P}(A)$ be the collection of subsets of $A$.
    Then there exists an ordinal $\alpha$, such that no no mapping $f_s: s \to \alpha$ from any subset $s \in S$ of $A$ to $\alpha$ is an order isomorphisms.
\end{lemma}

\begin{proof}\cite[Lemma]{Haj72}
    We let the ordinal $\alpha$ take the following value:
    $$\alpha = \cup\left\{ \text{type}\left(X, R\right) + 1 \,\vert\, X \subseteq A, R \subseteq A \times A \wedge R\ \text{well-orders}\ A\right\}.$$
    We will show that $\alpha$ is such that it satisfies the lemma's statement.

    % -> show that alpha is a set
    % -> show that alpha is an ordinal (so it is well-ordered)
\end{proof}

\end{document}