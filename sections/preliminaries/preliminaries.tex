\documentclass[../../main.tex]{subfiles}
\begin{document}
The convention in this thesis will be to say \textbf{ZF} when talking about Zermelo-Fraenkel set theory \textit{without} the axiom of choice.
When talking about the axiom of choice on its own we will say \textbf{AC}, and when talking about Zermelo-Fraenkel set theory together with the axiom of choice we use \textbf{ZFC}.

We will use the convention of including $0$ at the beginning of the natural numbers $\mathbb{N}$, i.e. $\mathbb{N} = \left\{0, \, 1, \, 2\, \ldots\right\}$.
This is a \textit{natural} choice, since we then can use $\mathbb{N}$ to mean the set described by the \hyperref[ZF7]{Axiom of Infinity}.

\section{First-order Logic and Classes}

\section{Zermelo-Fraenkel Axioms of Set Theory}
We assume that the reader has some familiarity with axiomatic set theory, but for convenience and consistency we restate some of the necessary basics here. %change wording
For a more thorough introduction of the topic, see \cite[4.3-4.5]{Gol17}, alternatively \cite[1.1]{Jec78} gives a more technical overview.
The formulation of the axioms below is based on both textbooks.
%Double check that the formulas used are not identical to Goldrei

\subsection{Axiom of Extensionality}
$$\forall x \forall y \left(x = y \iff \forall z \left(z \in x \iff z \in y\right) \right)$$
Two sets are equal if and only if they contain the same elements.\cite[4.3, p.76]{Gol17}

\subsection{Axiom of Pairs}
$$\forall x \forall y \exists z \forall w \left( w \in z \iff \left(w = x \vee w = y \right)\right)$$
For any two sets, there is a set whose elements are precisely these sets.
 
We define an ordered pair $\left<x,\, y\right>$ to be the set $\left\{\left\{x\right\},\, \left\{x,\, y\right\}\right\}$.
Further, ordered $n$-tuples are defined recursively as $\left<x_1,\, x_2,\, x_3,\ldots,\, x_n\right> = \left<x_1,\, \left<x_2,\, x_3,\ldots,\, x_n\right>\right>$.\cite[4.3, pp.76, 79-80]{Gol17}

Ordered pairs satisfy the property that for any sets $x$, $y$, $u$, $v$, if $\left<x,\, y\right> = \left<u,\, v\right>$, then $x = u$ and $y = v$.\cite[Theorem 4.2, p.79]{Gol17}

\subsection{Axiom Schema of Separation}\label{ZF3}
Let $\phi(z, p)$ be a formula in first order logic with a free variable $z$. Then
\begin{equation}\label{seperation-axiom}
    \forall x \forall p \exists y \forall z \left(z \in y \iff \left(z \in x \wedge \phi(z, p)\right)\right).
\end{equation}
For any sets $x$ and $p$ there is a unique set consisting of all $z$ in $x$ for which $\phi(z, p)$ holds.
This is an axiom schema, meaning an infinite collection of axioms, since (\ref{seperation-axiom}) is a separate axiom for every formula $\phi(z, p)$.\cite[pp.5-6]{Jec78}
%able to define intersections between a class and a set as a set itself

\subsection{Axiom of the Empty Set} 
$$\exists x \forall y \ y \notin x$$
There is a set with no elements. We call this set $\varnothing = \left\{\right\}$. \cite{Gol17}

The Empty Set Axiom is not strictly required, the existence of the empty set also arises from the \hyperref[ZF3]{Axiom Schema of Seperation}.
Since we can define the empty class $\varnothing = \left\{u\, \vert\, u \neq u\right\}$, the empty set is also a set.
However this follows from $\varnothing$ being a subset of all sets and hence only under the assumption that at least one other set exists.
The existence of that set, in turn, follows from the \hyperref[ZF7]{Axiom of Infinity}.\cite[p.6]{Jec78}

\subsection{Axiom of Power Sets}
$$\forall x \exists y \forall z \left(z \in y \iff z \subseteq x\right)$$
For any set $x$ there is a set, denoted by $\mathcal{P}(x)$ and called the power set of x, consisting of all subsets of x.
%cartesian products and functions

\subsection{Union Axiom}
$$\forall x \exists y \forall z \left(z \in y \iff \exists w \left(z \in w \wedge w \in x\right)\right)$$
For any set $x$ there is a set, denoted by $\bigcup x$, which is the union of all the elements of $x$.

\subsection{Axiom of Infinity} \label{ZF7}
$$\exists x \left(\varnothing \in x \wedge \forall y \left(y \in x \implies y \cup \{y\} \in x \right)\right)$$
There is an inductive set.

\subsection{Axiom of Replacement}
$$\forall x \exists y \forall y' \left(y' \in y \iff \exists x' \left(x' \in x \wedge \phi(x', y')\right)\right),$$
where $\phi(s, t)$ is a formula such that 
$$\forall s \exists t \left(\phi(s, t) \wedge \forall t' \left(\phi(s, t') \implies t' = t\right)\right).$$
If $\phi(s, t)$ is a class function, then when its domain is restricted to a set $x$, the resulting images form a set $y$.
%rework to be axiom schema of replacement

\subsection{Axiom of Foundation}
$$\forall x \exists y \left(y \in x \wedge x \cap y = \varnothing \right)$$
Every set contains an $\in$-minimal element, we call this being \textit{well-founded}.\cite[p.92]{Gol17}
This also means there exist no infinitely descending chains of sets, such as $x_0 \ni x_1 \ni x_2 \ni \cdots$.\cite[Theorem 4.3, p.95]{Gol17}

\section{The Axiom of Choice}
%It has to be somewhere, right?

\end{document}