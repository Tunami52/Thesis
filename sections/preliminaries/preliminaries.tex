\documentclass[../../main.tex]{subfiles}
\begin{document}
The convention in this thesis will be to say \textbf{ZF} when talking about Zermelo-Fraenkel set theory \textit{without} the axoim of choice.
When talking about the axoim of choice on its own we will say \textbf{AC}, and when talking about Zermelo-Fraenkel set theory together with the axiom of choice use \textbf{ZFC}.

We will use the convention of including 0 at the beginning of the natural numbers $\mathbb{N}$, i.e. $\mathbb{N} = \left\{0, \, 1, \, 2\, \ldots\right\}$.
This is a \textit{natural} choice, since we then can use $\mathbb{N}$ to mean the set described by the \hyperref[ZF7]{Axiom of Infinity}.

\section{Zermelo-Fraenkel Axioms of Set Theory}
We assume that the reader has some familiarity with axiomatic set theory, but for convenience and constistency we restate some of the necessary basics here. %change wording
For a more thorough review, see \cite{Gol17}, from which the formulations below are used as well.
%Double check that the formulas used are not identical to Goldrei

\subsection{Axiom of Extensionality}
$$\forall x \forall y \left(x = y \iff \forall z \left(z \in x \iff z \in y\right) \right)$$
Two sets are equal if and only if they contain the same elements.

\subsection{Axiom of the Empty Set}
$$\exists x \forall y \ y \notin x$$
There is a set with no elements.

\subsection{Axiom of Pairs}
$$\forall x \forall y \exists z \forall w \left( w \in z \iff \left(w = x \vee w = y \right)\right)$$
For any two sets, there is a set whose elements are precisely these sets.

\subsection{Axiom of Seperation}
$$\forall x \exists y \forall z \left(z \in y \iff \left(z \in x \wedge \phi(z)\right)\right),$$
where $\phi(z)$ is any statement of the formal language with free variable $z$.
For any set $x$ there is a set consiting of all $z$ in $x$ for which $\phi(z)$ holds.

\subsection{Axiom of Power Sets}
$$\forall x \exists y \forall z \left(z \in y \iff z \subseteq x\right)$$
For any set $x$ there is a set, denoted by $\mathcal{P}(x)$ and called the power set of x, consisting of all subsets of x.

\subsection{Union Axiom}
$$\forall x \exists y \forall z \left(z \in y \iff \exists w \left(z \in w \wedge w \in x\right)\right)$$
For any set $x$ there is a set, denoted by $\bigcup x$, which is the union of all the elements of $x$.

\subsection{Axiom of Infinity} \label{ZF7}
$$\exists x \left(\varnothing \in x \wedge \forall y \left(y \in x \implies y \cup \{y\} \in x \right)\right)$$
There is an inductive set.

\subsection{Axiom of Replacement}
$$\forall x \exists y \forall y' \left(y' \in y \iff \exists x' \left(x' \in x \wedge \phi(x', y')\right)\right),$$
where $\phi(s, t)$ is a formula such that 
$$\forall s \exists t \left(\phi(s, t) \wedge \forall t' \left(\phi(s, t') \implies t' = t\right)\right).$$
If $\phi(s, t)$ is a class function, then when its domain is restricted to a set $x$, the resulting images form a set $y$.
Since we do not treat any 

\subsection{Axiom of Foundation}
$$\forall x \exists y \left(y \in x \wedge x \cap y = \varnothing \right)$$
Every set is \textit{well-founded}, i.e. contains an $\in$-minimal element.

\end{document}