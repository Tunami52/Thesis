\documentclass[../../main.tex]{subfiles}
\begin{document}
The convention in this thesis will be to say \textbf{ZF} when talking about Zermelo-Fraenkel set theory \textit{without} the axiom of choice.
When talking about the axiom of choice on its own we will say \textbf{AC}, and when talking about Zermelo-Fraenkel set theory together with the axiom of choice we use \textbf{ZFC}.

We will use the convention of including $0$ at the beginning of the natural numbers $\mathbb{N}$, i.e. $\mathbb{N} = \left\{0, \, 1, \, 2\, \ldots\right\}$.
This is a \textit{natural} choice, since we then can use $\mathbb{N}$ to mean the set described by the \hyperref[ZF7]{Axiom of Infinity}.
If we want to talk about strictly positive integers we use the notation $\mathbb{Z}^+ = \mathbb{N}\setminus\left\{0\right\}$.

Lastly, whenever we deal with the negation of some symbol, we cross it out to mean this, for example $a \neq b$ means $\lnot \left(a = b\right)$.

\section{First-Order Logic and Classes}
When talking about first order logic we mean the symbol $$= \text{ (equals)}$$ in conjunction with the logical connectives
$$\lnot \text{ (not), } \wedge \text{(and), } \vee \text{(or), } \rightarrow \text{(implies) and } \leftrightarrow \text{(if and only if)}$$
as well as the quantifiers $$\forall \text{ (for all) and } \exists \text{ (exists)}.$$
Additionally, as we are talking about sets, we use the symbol $\in$ to denote set inclusion. \cite[pp.2-3]{Jec78}

In order to effectively talk about properties of set and set-like structures we need to somehow properly define formulas.
We will go more into depth about formulas in Chapter \ref{model-theory}, where we will introduce the notion of a formal language.

For now we are only concerned with the language of sets defined above.

\begin{definition}[Formula of a Set]
    An \textit{atomic formula} in set theory is either
    \begin{enumerate}
        \item $x = y$, or
        \item $x \in y$
    \end{enumerate}
    A \textit{formula} $\phi$ is an any combination of atomic formulas with logical connectives and quantifiers.
\end{definition}

The symbols $x$ and $y$ above are called variables and for any two variables an atomic formula is either true or false for each $x$ and $y$.
A variable \textit{occurs freely} inside of a formula if it does not appear inside of a $\exists$ or $\forall$ quantifier, otherwise the variable is \textit{bound}.
We write $\phi\left(x_1,\ldots,\, x_n\right)$ for a formula with $n \in \mathbb{Z}^+$ free variables.
A formula where every variable is bound is called a \textit{sentence}. \cite[pp.10-11]{Mar02}

A sentence is either true are false and a formula with free variables is true or false for each choice of the free variables.
Each of the \textbf{ZF} axioms below are examples of sentences and as axioms we assume them to be inherently true (within the framework of our theory).
An example of a formula with a free variable would be $$\phi(x) = \exists y \, \left(y \in x\right).$$ 
This formula is only false for the empty set, since it is the unique set which does not contain any elements.
In that sense we think of formulas with free variables describing a \textit{property}, something we make use of in \textit{classes}.

\begin{definition}[Class]
    Let $\phi\left(x,\, p_1,\ldots,\, p_n\right)$ be a formula in first order logic.
    Then a \textit{class} $\mathbf{C}$ is defined as
    $$\mathbf{C} = \left\{x \,\vert\, \phi\left(x,\, p_1,\ldots,\, p_n\right)\right\}.$$
    The class $\mathbf{C}$ is called \textit{definable from} $p_1,\ldots,\, p_n$.
    Furthermore, if $x$ is the only free variable of $\phi$, the class $\mathbf{C}$ is simply called \textit{definable}. \cite[p.3]{Jec78}
\end{definition}

In practice, we use classes as a tool to help us construct useful sets in \textbf{ZFC}, as elements of classes are always sets in the stricter sense.
All sets are classes, but not all classes are sets, since if we have a fixed set $s$ we can always construct the corresponding class would be $\mathbf{S} = \left\{x \,\vert\, x = s\right\}$.
A class which is not a set is called a \textit{proper class}.

We consider to two classes to be the same if they have the same elements.
The familiar operations of \textit{inclusion}, \textit{union}, \textit{intersection}, and \textit{difference} are definable using formulas.
As such for classes $\mathbf{C}$, $\mathbf{D}$,
\begin{align*}
    \mathbf{C} \subseteq \mathbf{D} &\iff \forall x \left(x \in \mathbf{C} \implies x \in \mathbf{D}\right) \\
    \mathbf{C} \cup \mathbf{D} &= \left\{x \,\vert\, x \in \mathbf{C} \vee x \in \mathbf{D}\right\} \\
    \mathbf{C} \cap \mathbf{D} &= \left\{x \,\vert\, x \in \mathbf{C} \wedge x \in \mathbf{D}\right\} \\ 
    \mathbf{C} \setminus \mathbf{D} &= \left\{x \,\vert\, x \in \mathbf{C} \wedge x \not\in \mathbf{D}\right\} \\
    \bigcup \mathbf{C} &= \left\{x \,\vert\, x \in S \ \text{for some}\ S \in \mathbf{C}\right\}
\end{align*}\cite[pp.3-4]{Jec78}

For the use in this text, classes are in a sense ``naive sets-like object''; they to help us describe collections of sets without worrying about paradoxes.
Consider for example the class $\mathbf{V} = \left\{x \,\vert\, x=x\right\}$, which is the universe of all sets and does not exist in pure set theory.
Another important class which we will make use of later is the \textit{empty class} $\varnothing = \left\{x \,\vert\, x \neq x\right\}$ (although this is also a set as we will see).

\section{Zermelo-Fraenkel Axioms of Set Theory}
We assume that the reader has some familiarity with axiomatic set theory, but for convenience and consistency we restate some of the necessary basics here. %change wording
For a more thorough introduction of the topic, see \cite[\S\S 4.3-4.5]{Gol17}, alternatively \cite[\S 1.1]{Jec78} gives a more technical overview.
The formulation of the axioms below is based on both textbooks.
%Double check that the formulas used are not identical to Goldrei

\subsection{Axiom of Extensionality}\label{ZF1}
$$\forall x\, \forall y\, \left(x = y \iff \forall z \left(z \in x \iff z \in y\right) \right)$$
Two sets are equal if and only if they contain the same elements.\cite[\S 4.3, p.76]{Gol17}

\subsection{Axiom of Pairs}\label{ZF2}
$$\forall x\, \forall y\, \exists z\, \forall w\, \left( w \in z \iff \left(w = x \vee w = y \right)\right)$$
For any two sets, there is a set whose elements are precisely these sets.
 
We define an ordered pair $\left<x,\, y\right>$ to be the set $\left\{\left\{x\right\},\, \left\{x,\, y\right\}\right\}$.
Further, ordered $n$-tuples are defined recursively as $\left<x_1,\, x_2,\, x_3,\ldots,\, x_n\right> = \left<x_1,\, \left<x_2,\, x_3,\ldots,\, x_n\right>\right>$.\cite[\S 4.3, pp.76, 79-80]{Gol17}

Ordered pairs satisfy the property that for any sets $x$, $y$, $u$, $v$, if $\left<x,\, y\right> = \left<u,\, v\right>$, then $x = u$ and $y = v$.\cite[Theorem 4.2]{Gol17}

\subsection{Axiom Schema of Separation}\label{ZF3}
Let $\phi(z,\ p)$ be a formula in first order logic with free variables $z$ and $x$. Then
\begin{equation}\label{seperation-axiom}
    \forall x\, \forall p\, \exists y\, \forall z\, \left(z \in y \iff \left(z \in x \wedge \phi(z,\, p)\right)\right).
\end{equation}
For any sets $x$ and $p$ there exists a unique set consisting of all elements $z$ in $x$ for which $\phi(z,\, p)$ holds.
This is an axiom schema, meaning an infinite collection of axioms, since (\ref{seperation-axiom}) is a separate axiom for every formula $\phi(z,\, p)$.

Assume $\psi(z,\, p_1,\ldots,\, p_n)$ is a more general formula for which we want to utilize the axiom schema for.
We can then let $\phi(z,\ p)$ be the formula
$$\phi(z,\, p) = \exists p_1 \cdots \exists p_n\, \left(\left(p = \left<p_1,\ldots,\, p_n\right>\right) \ \text{and} \ \psi(z,\, p_1,\ldots,\, p_n)\right),$$
where we have that $\phi(z,\, \left<p_1,\ldots,\, p_n\right>)$ is true if and only if $\psi(z,\, p_1,\ldots,\, p_n)$ is true.
Hence we can generalize the axiom schema to 
\begin{equation}\label{general-seperation-axiom}
    \forall x\, \forall p_1 \cdots \forall p_n\, \exists y\, \forall z\, \left(z \in y \iff \left(z \in x \wedge \psi(z,\, p_1,\ldots,\, p_n)\right)\right).
\end{equation}

Let $\mathbf{C} = \left\{z \,\vert\, \psi(z,\, p_1,\ldots,\, p_n)\right\}$ be the class containing all sets $z$ which satisfy the formula $\psi$ for any free variables $p_1,\ldots,\, p_n$.
We can then utilize \ref{general-seperation-axiom} and have that
$$\forall x \, \exists y \, \left(\mathbf{C} \cap x = y\right),$$
describes the same set.
This means that any subclass of a set is also a set, and naturally a subclass of a set is called a subset. 

As a consequence the set theoretic operations \textit{difference} $\left(x \setminus y\right) \subseteq x$ and \textit{intersection} $\left(x \cap y\right) \subseteq x$ are also defined for any two sets $x$ and $y$,
since all sets also are classes.
Further the intersection 
$$\bigcap \mathbf{C} = \left\{z \,\vert\, z \in x \ \text{for every}\ x \in \mathbf{C}\right\}$$
of a class $\mathbf{C}$ is a set, since it is a subset of all of its elements, which are strictly sets. \cite[pp.5-6]{Jec78}

\subsection{The Empty Set}\label{ZF4}
$$\exists x \forall y \ y \notin x$$
There is a set with no elements. We call this set $\varnothing = \left\{\right\}$. \cite{Gol17}

This statement stands out, as it is not an axiom, the existence of the empty set also arises from the \hyperref[ZF3]{Axiom Schema of Seperation}.
We include it here for the sake of completeness.

Since we can define the empty class $\varnothing = \left\{u\, \vert\, u \neq u\right\}$, the empty set is also a set.
However this follows from $\varnothing$ being a subset of all sets and hence only under the assumption that at least one other set exists.
The existence of that set, in turn, follows from the \hyperref[ZF7]{Axiom of Infinity}.\cite[p.6]{Jec78}

\subsection{Axiom of Unions}\label{ZF5}
$$\forall x \exists y \forall z \left(z \in y \iff \exists w \left(z \in w \wedge w \in x\right)\right)$$
For any set $x$ there is a set, denoted by $\bigcup x$, which is the union of all the elements of $x$.
meaning it contains all elements of the members of $x$.

For any sets, we recursively define their union as
\begin{align*}
    x \cup y &= \bigcup \left\{x,\, y\right\} \\
    x \cup y \cup z &= \left(x \cup y\right) \cup z \\
    &\cdots
\end{align*}
as well as a set with more then two members as,
\begin{align*}
    \left\{x_1,\ldots,\ x_n\right\} &= \left\{x_1\right\} \cup \cdots \cup \left\{x_n\right\}
\end{align*}
where the existence of $\left\{x,\, y\right\}$ and $\left\{x\right\} = \left\{x,\, x\right\}$ is justified by the \hyperref[ZF2]{Axiom of Pairs}. \cite[p.6]{Jec78}

\subsection{Axiom of Power Sets}\label{ZF6}
$$\forall x\, \exists y\, \forall z\, \left(z \in y \iff z \subseteq x\right)$$
For any set $x$ there is a set, denoted by $\mathcal{P}(x)$ and called the power set of $x$, consisting of all subsets $s \subseteq x$.

Using the axioms of \hyperref[ZF3]{Separation}, \hyperref[ZF5]{Union} and \hyperref[ZF6]{Power Set} we can define the \textit{cartesian product} of the sets $X$ and $X$ as
$$x \times y = \left\{(u,\, v) \,\vert\, u \in x \vee v \in y\right\} \subseteq \mathcal{P}\left(\mathcal{P}\left(x \cup y\right)\right),$$
and for multiple sets $x_1,\ldots,\, x_n$ as
$$x^n = \underbrace{x_1 \times \cdots x_{n-1} \times x_n}_{n\ \text{times}}  = \left(x_1 \times \cdots \times x_{n-1}\right) \times x_n.$$

We call a set $R \subseteq X^n$ a \textit{n-ary} relation over $X$.
In general $R$ is a set of tuples and if $R$ is a binary relation, we write $x,\ R\, y$ for $(x,\, y) \in R$.

A binary relation $f \subseteq X \times Y$ is called a \textit{function} (or \textit{map}), if
$$\left((x,\, y) \in f \wedge (x,\, z) \in f\right) \implies y = z$$
holds for all $x \in X$ and $y,\, z \in Y$, and if for all $u \in X$ there exists some $v \in Y$, such that $(u,\, v) \in f$.

In the definition above, we call $X$ the \textit{domain} and $Y$ the \textit{codomain} of $f$.
The set
$$\mathbf{Im}(f) = \left\{y \,\vert\, \exists x \in X \, \left((x,\, y) \in f\right)\right\}$$
is called the \textit{image} of $f$. 
In general we write $f: X \to Y$ for $f \subseteq X \times Y$ and $f(x) = y$ whenever $(x,\, y) \in f$, in the latter case saying that $x$ \textit{maps to} $y$ in $f$.

A function $f: X \to Y$ is called a \textit{surjection} if $\mathbf{Im}(f) = Y$
and an \textit{injection}, if $$\left(f(x_1) = y \wedge f(x_2) = y\right) \implies x_1 = x_2.$$ 
A function is \textit{bijection}, if it is both a surjection and an injection. \cite[pp.7-10]{Jec78}

\subsection{Axiom of Infinity} \label{ZF7}
$$\exists x \left(\varnothing \in x \wedge \forall y \left(y \in x \implies y \cup \{y\} \in x \right)\right)$$
There is an inductive set.
An inductive set contains both the empty set $\varnothing$ as well as the successor $x^+$ of every $x$ in the set.
In this context the successor of a set $x$ is defined as $x^+ = x \cup \left\{x\right\}$.
We will go into more detail on this in Section \ref{ordinals-section}.

\subsection{Axiom Schema of Replacement}
Let $\phi(x,\, y,\, p)$ a formula in first-order logic with free variables $x$, $y$ and $p$.
\begin{align}\label{replacement-axiom}
    &\forall x\, \forall y\, \forall z\, \left(\phi(x,\, y,\, p) \wedge \phi(x,\, z,\, p) \implies y = z\right)\nonumber \\
    &\implies \forall x'\, \exists y'\, \forall y\, \left(y \in y'\ \iff \left(\exists x \in x'\right) \phi(x,\, y,\, p)\right)
\end{align}

Similar to \hyperref[ZF3]{the Axiom Schema of Seperation}, this is an axiom schema, meaning that (\ref{replacement-axiom}) is a separate axiom for each formula $\phi(x,\, y,\, p)$.
We can also generalize the Axiom of replacement in a similar way, by replacing $\phi(x,\, y,\, p)$ with $\phi(x,\, y,\, p_1,\ldots,\, p_n)$.

The axiom schema states that if $\mathbf{F} = \left\{(x\, y) \,\vert\, \phi(x,\, y,\, p)\right\}$ is a class function,
then the image $\mathbf{Im}(\mathbf{F})$ is a set whenever we restrict the domain of $\mathbf{F}$ to a set.
Consequently this restriction of $\mathbf{F}$ is also a function of sets. \cite[p.11]{Jec78}

\subsection{Axiom of Foundation}
$$\forall x \exists y \left(y \in x \wedge x \cap y = \varnothing \right)$$
Every set contains an $\in$-minimal element, we call this being \textit{well-founded}.\cite[p.92]{Gol17}
This also means there exist no infinitely descending chains of sets, such as $x_0 \ni x_1 \ni x_2 \ni \cdots$.\cite[Theorem 4.3, p.95]{Gol17}

\section{The Axiom of Choice}
To talk about the axiom of choice we need to first define what a choice function is, the concept which the axiom is centered around.
\begin{definition}[Choice Function]\cite[p.38]{Jec78}
    Let $S$ be a family of nonempty sets.
    A function $f: S \to \bigcup S$ is called a \textit{choice function} of $S$ if
    $$f(X) \in X$$
    holds for all sets $X \in S$.
\end{definition}
The Axiom of Choice is then defined as follows:
\begin{definition}[Axiom of Choice]\cite[p.38]{Jec78}
    There exists a choice function for every family of nonempty sets.
\end{definition}

The Axiom of Choice is not always needed for showing that a choice function exist.
Take for example $S = \mathcal{P}(\mathbb{N})$, under the usual order $<$ every subset of $\mathbb{N}$ has a least element.
We can therefore construct a choice function $f: S \to \mathbb{N}$ by letting $f(N)$ be the unique least element of $N$ for $N \in S$.
This is however not possible for a family of possibly infinite subsets of $\mathbb{R}$; for example the open interval $\left(0,\, 1\right)$ does not contain a least element.

In general there does not always exist an external structure for sets which we can utilize to construct a choice function.
The Axiom of Choice ensures that we can, but not how that choice function might look like.
In fact \textbf{AC} is the only axiom of \textbf{ZFC} which states the existence of a mathematical object without explicitly defining it.
This is a powerful tool, but can lead lead to fairly unintuitive results. 
As such, with \textbf{AC} there exists a way to order the real numbers where every subset has a least element (including open intervals like $\left(0,\, 1\right)$)!
\end{document}