\documentclass[../../main.tex]{subfiles}
\begin{document}
In mathematics we always assume that we have something called a set of axioms, 
which kind of act like a starting point for all of the theory which is built up after.
Axioms are in essence something which we have to assume is true, because it is impossible to prove something from nothing.
For example. we assume that two sets of objects are the same if they contain exactly the same objects.

Modern mathematics, then, depends (mostly) on the so-called Zermelo-Fraenkel Axioms of set theory.
This is a special set of seven axioms that all other mathematics is developed from.
One of these axioms, the ``Axiom of Choice'', is of special interest to us.
It assures us that some mathematical objects exist without giving the specifics on how to construct them.
The Axiom of Choice in essence tells us that a certain type of function, a choice function, always exists.

Imagine you are in a library where all of the books are mixed up, there is no system to any of the placements and nothing is where you would expect it to be.
Consider now that you are given a list of all of the authors by whom the books in the library were written.
Your task is now to pick out exactly one book by each author.

This sounds possible but tedious, right? 
You can just go through all books one-by-one and whenever you encounter a book by a new author for the first time you put that book aside.
Soon enough you will have gone all of the books and in the pile of books you put aside one book by each author will be featured.

In mathematics it unfortunately is not always as easy.
This is because we can also consider infinite objects, in our analogy this would be a library with infinitely many books.

What if one author was very productive and wrote infinitely many books?
When going through a list of all books you could just get stuck on searching through works by that one author and never get to any other books.
Or maybe you have a list of infinitely many authors each having written infinitely many books.
Is there really a way to arrange the books in such a way that you could, even given infinitely much time, eventually pick out exactly one book by each author?

The Axiom of Choice assures us that you can always find a way to pick out one book by each author, 
no matter how many books there are in the library or how weird the list of authors gets.
This is in essence what a choice function is, a function which given an author always returns a book by that author.
The Axiom of Choice does not tell you anything about what strategy you should use however, just that such a strategy exists.

The Axiom of Choice is special in other regards too.
Namely, it is famous for its many equivalent formulations, the one we are interested in has to do with groups.

A group is a set of objects together with an operation that satisfies certain properties.
What is important here is that you can always ``smash'' two objects together and still get out another object of that same set.
Also, whenever you have an equation with one object and the result, you are always able infer what the other object in the operation originally was.
(Think solving for $x$ in $2 \cdot x = 4$.)

Take the set of whole numbers $\left\{\ldots,\, -1,\, 0,\, 1,\, 2,\ldots\right\}$ together with addition, ``$+$'', for example, this would be a group.
In the context of the infinite library from before, a group structure would maybe take the form of an index.
Whenever you pick out two books you then can look up that pair in the index and it will give you the name and shelf location of a some new book.

Without going into too much detail (yet), this thesis is focussed on the proving following equivalence:

Saying that there is a way to define an operation for any arbitrary set such that it is a group is equivalent to saying that the Axiom of Choice is true.
So if you are always able to define a group operation on a set you are also able to always define a choice ``book picking'' function.
Vise versa, if it is possible for you to create a choice function for all possible sets, 
it is also possible for you to make a group out of every possible set.
\end{document}