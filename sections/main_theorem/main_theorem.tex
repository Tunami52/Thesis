\documentclass[../../main.tex]{subfiles}
\begin{document}
\begin{definition}[Groupoid]
    Let $G$ be a set.
    We say that $G$ together with a binary operation $*$ is a \emph{(left) cancellative groupoid}, if 
    \begin{enumerate}[label=(\roman*)]
        \item \label{groupoid-axiom-1} for all $g,\, h \in G$ we have $g * h \in G$
        \item \label{groupoid-axiom-2} for all $g,\, h,\, k \in G$ we have $\left(g * h = k * h\right) \rightarrow g = k$
    \end{enumerate}
    Analogously $\left(G,\, *\right)$ is called right cancellative if we replace condition \ref{groupoid-axiom-2} by
    $\forall g,\, h,\, k \in G \ \left(g * h = g * k\right) \rightarrow h = k$.

    Let $\mathcal{L} = \left\{*\right\}$ be the language containing only the binary function $*$.
    Then an $\mathcal{L}$-structure $M$ is a model of a cancellative groupoid, 
    if and only if it satisfies the theory given by conditions \ref{groupoid-axiom-1}-\ref{groupoid-axiom-2}.
\end{definition}
An example of a cancellative groupoid (without identity and which is not a group) is the positive integers $\mathbb{Z}^+$ under addition.

We now finally arrive at our main theorem.
Note that in the proof of Theorem \ref{main-thm} it is not important weather a groupoid is left or right cancellative,
but rather that it has some form of the cancellation property.
In our case we choose to utilize a right cancellative groupoid.

\begin{theorem}\cite{Haj72}\label{main-thm}
    The following are equivalent in ZF:
    \begin{enumerate}
        \item The Axiom of Choice
        \item Every non-empty set admits a cancellative groupoid structure
    \end{enumerate}
\end{theorem}

\begin{proof}
    The theorem is proven in two steps, deriving a single direction implication for each sentence.
    \begin{enumerate}
        \item \textbf{Groupoid Structure on arbitrary sets} $\implies$ \textbf{Axiom of Choice}
        
        We show that the existence of a groupoid structure on every non-empty set implies that every set can be well-ordered.
        By Theorem \ref{well-order-thm-equivalence} this is equivalent to the Axiom of Choice.

        Let $A$ be an arbitrary set and let $\alpha$ be an ordinal as described in Theorem \ref{hartogs-lemma} in Section \ref{hartogs-lemma-section}.
        This means that there exists no bijective mapping from $\alpha$ to any subset of $A$ (including $A$ itself).
        We then let $\left(B,\, R\right)$ be a well-ordered set of type $\alpha$ and such that $A \cap B = \varnothing$.
        
        Now let $C$ be the set $C = A \cup B$, by assumption there exists some operation $+$, such that $\left(C,\ +\right)$ is a cancellative groupoid. %change notation!!
        We will show that for every $x \in A$ there exists $y \in B$, such that $x + y \in B$ holds.

        Let us assume for a contradiction that the above claim does not hold.
        This would imply that some $a \in A$ exists for which $a + y \in A$ holds for all $y \in B$.
        Let $f: B \to A$ be the function defined by $f(y) = a + y$.
        We have that $+$ is a cancellative groupoid operation, hence $f$ must be injective;
        a contradiction by Theorem \ref{hartogs-lemma}, since we had assumed that $B$ is of type $\alpha$.

        We let $D = B \times B$ be the well-ordered set with respect to the lexicographical ordering $R'$ of $R$,
        and define a function $g: A \to D$ by $$g(x) = \min_{R'} \left\{\left<u,\, v\right> \in D \,\vert\, x + u = v\right\}.$$
        
        The function $g$ maps every element $x$ of $A$ to the least pair $\left<u,\, v\right>$ in $B \times B$ satisfying $x + u = v$.
        From earlier in the proof we know that such a pair must exists and that $g$ must in fact be injective.
        This again follows from $+$ being cancellative, since if $x_1,\, x_2$ are two elements of $A$, having $f(x_1) = f(x_2)$ would imply that
        \begin{alignat*}{3}
                        x_1 + u  &= \quad \ v     &=& \ x_2 + u \\
            \iff \qquad x_1      &= \ v + u^{-1}\ &=& \ x_2     
        \end{alignat*} 
        for some pair $\left<u,\, v\right> \in D$.
        Since $\mathbf{Im}(g)$ is a subset of $D$ it itself is a well-ordered set.
        As such we can define a well-order $R''$ on $A$ by letting $x_i R'' x_j$ whenever $g(x_i) R' g(x_j)$.

        \item \textbf{Axiom of Choice} $\implies$ \textbf{Groupoid Structure on arbitrary sets}
        
        Let $A$ be a finite set with $n$ elements.
        Then we can imbue $A$ with the structure of the cyclic group $\mathbb{Z}_n$, since every group is also a cancellative groupoid.
        Similarly, if $A$ is countably infinite there exist a bijection to the set of integers $\mathbb{Z}$.
        Hence we can let $A$ have the group structure $\left(\mathbb{Z}, +\right)$ of integers under addition.

        Finally, since there exists a countably infinite model of a cancellative groupoid, we can apply Theorem \ref{up-lowenheim-skolem}.
        By this there also exist models of arbitrary infinite powers.

        Therefore it is possible to define a cancellative groupoid structure on every arbitrary set $A$. \qedhere
    \end{enumerate}
\end{proof}

\end{document}