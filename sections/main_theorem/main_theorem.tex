\documentclass[../../main.tex]{subfiles}
\begin{document}
We now arrive at our main theorem:

\begin{theorem}\cite{Haj72}\label{main-thm}
    The following are equivalent in ZF:
    \begin{enumerate}
        \item Axiom of Choice
        \item Every non-empty set admits a cancellative groupoid structure
    \end{enumerate}
\end{theorem}

\begin{proof}
    The theorem is proven in two steps, deriving a single direction implication for each sentence.
    \begin{enumerate}
        \item \textbf{Groupoid Structure on arbitrary Sets} $\implies$ \textbf{Axiom of Choice}
        
        We show that the existence of a groupoid structure on every non-empty set implies that every set can be well-ordered.
        By Theorem \ref{well-order-thm-equivalence} this is equivalent to the Axiom of Choice.

        Let $A$ be an arbitrary set and let $\alpha$ be an ordinal as described in Theorem \ref{hartogs-lemma} in Section \ref{hartogs-lemma-section}.
        This means that there exists no bijective mapping from $\alpha$ to any subset of $A$ (including $A$ itself).
        We then let $\left(B,\, R\right)$ be a well-ordered set of type $\alpha$ and such that $A \cap B = \varnothing$.
        
        Now let $C$ be the set $C = A \cup B$, by assumption there exists some operation $+$, such that $\left(C,\ +\right)$ is a cancellative groupoid. %change notation!!
        We will show that for every $x \in A$ there exists $y \in B$, such that $x + y \in B$ holds.

        Let us assume for a contradiction that the above claim does not hold.
        This would imply that some $a \in A$ exists for which $a + y \in A$ holds for all $y \in B$.
        Let $f: B \to A$ be the function defined by $f(y) = a + y$.
        We have that $+$ is a cancellative groupoid operation, hence $f$ must be injective;
        a contradiction by Theorem \ref{hartogs-lemma}, since we had assumed that $B$ is of type $\alpha$.

        We let $D = B \times B$ be the well-ordered set with respect to the lexicographical ordering $R'$ of $R$,
        and define a function $g: A \to D$ by $$g(x) = \min_{R'} \left\{\left<u,\, v\right> \in D \,\vert\, x + u = v\right\}.$$
        
        The function $g$ maps every element $x$ of $A$ to the least pair $\left<u,\, v\right>$ in $B \times B$ satisfying $x + u = v$.
        From earlier in the proof we know that such a pair must exists and that $g$ must in fact be injective.
        This again follows from $+$ being cancellative, since if $x_1,\, x_2$ are two elements of $A$, having $f(x_1) = f(x_2)$ would imply that
        \begin{align*}
                & x_1 + u = v          = x_2 + u &\\
           \iff & x_1     = v + u^{-1} = x_2     &
        \end{align*} 
        for some pair $\left<u,\, v\right> \in D$.
        Since $\mathbf{Im}(g)$ is a subset of $D$ it itself is a well-ordered set.
        As such we can define a well-order $R''$ on $A$ by letting $x_i R'' x_j$ whenever $g(x_i) R' g(x_j)$.

        \item \textbf{Axiom of Choice} $\implies$ \textbf{Groupoid Structure on arbitrary Sets}
        
        $\ldots$
    \end{enumerate}
\end{proof}

\end{document}